%me=0 student solutions (ps file), me=1 - my solutions (sol file), me=2 - assignment (hw file)
\def\me{0}
\def\num{1}  %homework number
\def\due{Thursday, February 11}  %due date
\def\course{DS-GA.1008 Deep Learning} %course name, changed only once
\def\name{R2DEEP2 (Ankit Vani, Srivas Venkatesh)}   %student changes (instructor keeps!)
%
\iffalse
INSTRUCTIONS: replace # by the homework number.
(if this is not ps#.tex, use the right file name)

  Clip out the ********* INSERT HERE ********* bits below and insert
appropriate TeX code.  Once you are done with your file, run

  ``latex ps#.tex''

from a UNIX prompt.  If your LaTeX code is clean, the latex will exit
back to a prompt.  To see intermediate results, type

  ``xdvi ps#.dvi'' (from UNIX prompt)
  ``yap ps#.dvi'' (if using MikTex in Windows)

after compilation. Once you are done, run

  ``dvips ps#.dvi''

which should print your file to the nearest printer.  There will be
residual files called ps#.log, ps#.aux, and ps#.dvi.  All these can be
deleted, but do not delete ps1.tex. To generate postscript file ps#.ps,
run

  ``dvips -o ps#.ps ps#.dvi''

I assume you know how to print .ps files (``lpr -Pprinter ps#.ps'')
\fi
%
\title{Deep Learning}
\documentclass[11pt]{article}
\usepackage{amsfonts,amsmath,physics}
\usepackage{latexsym, graphicx}
\usepackage{amssymb}
\usepackage{mathtools}
\usepackage{clrscode3e}
\usepackage{tikz}
\usepackage{bm}
\usetikzlibrary{trees}
\usepackage{tikz-qtree}
\setlength{\oddsidemargin}{.0in}
\setlength{\evensidemargin}{.0in}
\setlength{\textwidth}{6.5in}
\setlength{\topmargin}{-0.4in}
\setlength{\textheight}{8.5in}

\newcommand{\handout}[5]{
   \renewcommand{\thepage}{#1, Page \arabic{page}}
   \noindent
   \begin{center}
   \framebox{
      \vbox{
    \hbox to 5.78in { {\bf \course} \hfill #2 }
       \vspace{4mm}
       \hbox to 5.78in { {\Large \hfill #5  \hfill} }
       \vspace{2mm}
       \hbox to 5.78in { {\it #3 \hfill #4} }
      }
   }
   \end{center}
   \vspace*{4mm}
}

\newcommand{\LCA}{\mbox{\sf LCA}}

\newcommand{\rs}{\rightsquigarrow}
\newcommand{\ls}{\leftsquigarrow}

\newcounter{pppp}
\newcommand{\prob}{\arabic{pppp}}  %problem number
\newcommand{\increase}{\addtocounter{pppp}{1}}  %problem number

%first argument desription, second number of points
\newcommand{\newproblem}[1]{
\ifnum\me=0
\ifnum\prob>0 \newpage \fi
\increase
\setcounter{page}{1}
\handout{\name, Assignment \num, Section \arabic{pppp}}{\today}{Team: \name}{Due:
\due}{Solutions to Assignment \num}
\section*{Problem \prob~ - #1 \hfill}
\else
\increase
\section*{Problem \num-\prob~ - #1 \hfill}
\fi
}

%\newcommand{\newproblem}[2]{\increase
%\section*{Problem \num-\prob~(#1) \hfill {#2}}
%}

\def\squarebox#1{\hbox to #1{\hfill\vbox to #1{\vfill}}}
\def\qed{\hspace*{\fill}
        \vbox{\hrule\hbox{\vrule\squarebox{.667em}\vrule}\hrule}}
\newenvironment{solution}{\begin{trivlist}\item[]{\bf Solution:}}
                      {\qed \end{trivlist}}
\newenvironment{solsketch}{\begin{trivlist}\item[]{\bf Solution Sketch:}}
                      {\qed \end{trivlist}}
\newenvironment{code}{\begin{tabbing}
12345\=12345\=12345\=12345\=12345\=12345\=12345\=12345\= \kill }
{\end{tabbing}}

%%%%%\newcommand{\eqref}[1]{Equation~(\ref{eq:#1})}

\newcommand{\hint}[1]{({\bf Hint}: {#1})}
%Put more macros here, as needed.
\newcommand{\room}{\medskip\ni}
\newcommand{\brak}[1]{\langle #1 \rangle}
\newcommand{\bit}[1]{\{0,1\}^{#1}}
\newcommand{\zo}{\{0,1\}}
\newcommand{\C}{{\cal C}}

\newcommand{\nin}{\not\in}
\newcommand{\set}[1]{\{#1\}}
\renewcommand{\ni}{\noindent}
\renewcommand{\gets}{\leftarrow}
\renewcommand{\to}{\rightarrow}
\newcommand{\assign}{:=}
\newcommand{\cT}{\mathcal{T}}

\DeclareMathOperator*{\E}{E}
\DeclareMathOperator*{\argmin}{arg\,min}
\DeclareMathOperator*{\argmax}{arg\,max}
\DeclareMathOperator*{\ORA}{\vee}
\newcommand{\R}{\mathbb{R}}

%\DeclarePairedDelimiter{\abs}{\lvert}{\rvert}
%\DeclarePairedDelimiter{\norm}{\lVert}{\rVert}
\DeclarePairedDelimiter{\inprod}{\langle}{\rangle}

\newcommand\Perm[2][n]{\prescript{#1\mkern-2.5mu}{}P_{#2}}

\newcommand{\AND}{\wedge}
\newcommand{\OR}{\vee}

\newcommand{\mexp}{\mathrm{e}}


\begin{document}


\newproblem{Backpropagation}


\begin{enumerate}

\item Warmup: Logistic regression is a pretty popular technique in machine learning to
classify data into two categories. This technique builds over linear regression by using
the same linear model but this is followed by the sigmoid function which converts
the output of the linear model to a value between 0 and 1. This value can then be
interpreted as a probability. This is usually represented as:
\begin{equation}
P(y=1|x_{in}) = x_{out} = \sigma(x_{in}) = \frac{1}{1+\mexp^{-x_{in}}}
\end{equation}
where $x_{in}$ as the name would suggest is the input scalar (which is also the output of
linear model) and $x_{out}$ is the output scalar.
\\If the error backpropagated to $x_{out}$ is $\pdv{E}{x_{out}}$
, write the expression for $\pdv{E}{x_{in}}$ in terms of $\pdv{E}{x_{out}}$.

\ifnum\me<2
\begin{solution}
\\We know that using the chain rule we can write $\pdv{E}{x_{in}} = \pdv{E}{x_{out}} \pdv{F(x_{in})}{x_{in}}$ where $F(x_{in}) = P(y=1|x_{in})$. Applying this to the above function we get:
\begin{equation}
\begin{aligned}
\pdv{F(x_{in})}{x_{in}} &= \frac{\mexp^2}{(1+\mexp^2)}\\
\implies \pdv{E}{x_{in}} &= \pdv{E}{x_{out}} \frac{\mexp^2}{(1+\mexp^2)}
\end{aligned}
\end{equation}
\end{solution}
\fi



\item Multinomial logistic regression is a generalization of logistic regression into multiple
classes. The softmax expression is at the crux of this technique. After receiving $n$
unconstrained values, the softmax expression normalizes these values to $n$ values that
all sum to 1. This can then be perceived as probabilities attributed to the various
classes by a classifier. Your task here is to backpropagate error through this module.
The softmax expression which indicates the probability of the $i$-th class is as follows:
\begin{equation}
P(y=i|X_{in}) = (X_{out})_i = \frac{\mexp^{-\beta (X_{in})_i}}{\sum_k \mexp^{-\beta (X_{in})_k}}
\end{equation}
What is the expression for $\pdv{(X_{out})_i}{(X_{in})_j}$? (Hint: Answer differs when $i = j$ and $i \neq j$).
\\The variables $X_{in}$ and $X_{out}$ aren’t scalars but vectors. While $X_{in}$ represents the $n$
values input to the system, $X_{out}$ represents the $n$ probabilities output from the system.
Therefore, the expression $(X_{out})_i$ represents the $i$-th element of $X_{out}$.
\ifnum\me<2
\begin{solution}
\\For this we will consider that components of $X_{in}$ are independent of one another. Then using simple product rule of differentiation we get:
\begin{equation}
\pdv{(X_{out})_i}{(X_{in})_j} =
\begin{cases}
-\beta \frac{\mexp^{-\beta (X_{in})_i}}{\sum_k \mexp^{-\beta (X_{in})_k}}
+ \beta \left( \frac{\mexp^{-\beta (X_{in})_i}}{\sum_k \mexp^{-\beta (X_{in})_k}} \right)^2
= \beta (X_{out})_i ((X_{out})_i -1)
&\text{if } i=j\\\\
\beta \frac{\mexp^{-\beta (X_{in})_i} \cdot \mexp^{-\beta (X_{in})_j}}{\left( \sum_k \mexp^{-\beta (X_{in})_k} \right)^2}
= \beta (X_{out})_i (X_{out})_j
&\text{if } i \neq j 
\end{cases}
\end{equation}
\end{solution}
\fi


\end{enumerate}



\newproblem{Torch (MNIST Handwritten Digit Recognition)}

\ifnum\me<2
\begin{itemize}
\item[]
\begin{solution}
TODO
\end{solution}

\fi

\end{itemize}



\end{document}


