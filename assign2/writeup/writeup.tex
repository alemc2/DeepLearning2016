%me=0 student solutions (ps file), me=1 - my solutions (sol file), me=2 - assignment (hw file)
\def\me{0}
\def\num{1}  %homework number
\def\due{Thursday, March 10}  %due date
\def\course{DS-GA.1008 Deep Learning} %course name, changed only once
\def\name{R2DEEP2 (Ankit Vani, Srivas Venkatesh)}   %student changes (instructor keeps!)
%
\iffalse
INSTRUCTIONS: replace # by the homework number.
(if this is not ps#.tex, use the right file name)

  Clip out the ********* INSERT HERE ********* bits below and insert
appropriate TeX code.  Once you are done with your file, run

  ``latex ps#.tex''

from a UNIX prompt.  If your LaTeX code is clean, the latex will exit
back to a prompt.  To see intermediate results, type

  ``xdvi ps#.dvi'' (from UNIX prompt)
  ``yap ps#.dvi'' (if using MikTex in Windows)

after compilation. Once you are done, run

  ``dvips ps#.dvi''

which should print your file to the nearest printer.  There will be
residual files called ps#.log, ps#.aux, and ps#.dvi.  All these can be
deleted, but do not delete ps1.tex. To generate postscript file ps#.ps,
run

  ``dvips -o ps#.ps ps#.dvi''

I assume you know how to print .ps files (``lpr -Pprinter ps#.ps'')
\fi
%
\title{Deep Learning}
\documentclass[11pt]{article}
\usepackage{amsfonts,amsmath,physics}
\usepackage[backend=bibtex,sorting=none]{biblatex}
\usepackage{latexsym, graphicx}
\usepackage{amssymb, gensymb}
\usepackage{mathtools}
\usepackage{clrscode3e}
\usepackage{longtable}
\usepackage{tikz}
\usepackage{bm}
\usepackage{hyperref}
\usetikzlibrary{trees}
\usepackage{tikz-qtree}
\usepackage{graphicx,float}
\setlength{\oddsidemargin}{.0in}
\setlength{\evensidemargin}{.0in}
\setlength{\textwidth}{6.5in}
\setlength{\topmargin}{-0.4in}
\setlength{\textheight}{8.5in}

\addbibresource{bibliography.bib}
\graphicspath{ {mnist/mistakeims/} }

\newcommand{\handout}[5]{
   \renewcommand{\thepage}{#1, Page \arabic{page}}
   \noindent
   \begin{center}
   \framebox{
      \vbox{
    \hbox to 5.78in { {\bf \course} \hfill #2 }
       \vspace{4mm}
       \hbox to 5.78in { {\Large \hfill #5  \hfill} }
       \vspace{2mm}
       \hbox to 5.78in { {\it #3 \hfill #4} }
      }
   }
   \end{center}
   \vspace*{4mm}
}

\newcommand{\LCA}{\mbox{\sf LCA}}

\newcommand{\rs}{\rightsquigarrow}
\newcommand{\ls}{\leftsquigarrow}

\newcounter{pppp}
\newcommand{\prob}{\arabic{pppp}}  %problem number
\newcommand{\increase}{\addtocounter{pppp}{1}}  %problem number

%first argument desription, second number of points
\newcommand{\newproblem}[1]{
\ifnum\me=0
\ifnum\prob>0 \newpage \fi
\increase
\setcounter{page}{1}
\handout{\name, Assignment \num, Section \arabic{pppp}}{\today}{Team: \name}{Due:
\due}{Solutions to Assignment \num}
\section*{Problem \prob~ - #1 \hfill}
\else
\increase
\section*{Problem \num-\prob~ - #1 \hfill}
\fi
}

%\newcommand{\newproblem}[2]{\increase
%\section*{Problem \num-\prob~(#1) \hfill {#2}}
%}

\def\squarebox#1{\hbox to #1{\hfill\vbox to #1{\vfill}}}
\def\qed{\hspace*{\fill}
        \vbox{\hrule\hbox{\vrule\squarebox{.667em}\vrule}\hrule}}
\newenvironment{solution}{\begin{trivlist}\item[]{\bf Solution:}}
                      {\qed \end{trivlist}}
\newenvironment{solsketch}{\begin{trivlist}\item[]{\bf Solution Sketch:}}
                      {\qed \end{trivlist}}
\newenvironment{code}{\begin{tabbing}
12345\=12345\=12345\=12345\=12345\=12345\=12345\=12345\= \kill }
{\end{tabbing}}

%%%%%\newcommand{\eqref}[1]{Equation~(\ref{eq:#1})}

\newcommand{\hint}[1]{({\bf Hint}: {#1})}
%Put more macros here, as needed.
\newcommand{\room}{\medskip\ni}
\newcommand{\brak}[1]{\langle #1 \rangle}
\newcommand{\bit}[1]{\{0,1\}^{#1}}
\newcommand{\zo}{\{0,1\}}
\newcommand{\C}{{\cal C}}

\newcommand{\nin}{\not\in}
\newcommand{\set}[1]{\{#1\}}
\renewcommand{\ni}{\noindent}
\renewcommand{\gets}{\leftarrow}
\renewcommand{\to}{\rightarrow}
\newcommand{\assign}{:=}
\newcommand{\cT}{\mathcal{T}}

\DeclareMathOperator*{\E}{E}
\DeclareMathOperator*{\argmin}{arg\,min}
\DeclareMathOperator*{\argmax}{arg\,max}
\DeclareMathOperator*{\ORA}{\vee}
\newcommand{\R}{\mathbb{R}}

%\DeclarePairedDelimiter{\abs}{\lvert}{\rvert}
%\DeclarePairedDelimiter{\norm}{\lVert}{\rVert}
\DeclarePairedDelimiter{\inprod}{\langle}{\rangle}

\newcommand\Perm[2][n]{\prescript{#1\mkern-2.5mu}{}P_{#2}}

\newcommand{\AND}{\wedge}
\newcommand{\OR}{\vee}

\newcommand{\mexp}{\mathrm{e}}

\makeatletter
\newtoks\@tabtoks
\newcommand\addtabtoks[1]{\global\@tabtoks\expandafter{\the\@tabtoks#1}}
\newcommand\eaddtabtoks[1]{\edef\mytmp{#1}\expandafter\addtabtoks\expandafter{\mytmp}}
\newcommand*\resettabtoks{\global\@tabtoks{}}
\newcommand*\printtabtoks{\the\@tabtoks}
\makeatother

\allowdisplaybreaks


\begin{document}


\newproblem{More Backpropagation}


\begin{enumerate}

\item[1.1] \textbf{Backpropagation through a DAG of modules}
\ifnum\me<2
\begin{solution}\\
Let $o_{max}$ be the output of the \texttt{max} module, and $o_{min}$ be the output of the \texttt{min} module.

Let $i_1$ be the output of the first \texttt{sigmoid} module, and $i_2$ be the output of the second \texttt{sigmoid} module. Furthermore, let $i = \begin{bmatrix} i_1 \\ i_2 \end{bmatrix}$.

We have:
\begin{align*}
\frac{\partial o_{max}}{\partial x_1} &= \frac{\partial o_{max}}{\partial i} \frac{\partial i}{\partial x_1}\\
&= \begin{bmatrix} \frac{\partial o_{max}}{\partial i_1} & \frac{\partial o_{max}}{\partial i_2} \end{bmatrix} \begin{bmatrix} \frac{\partial i_1}{\partial x_1} \\ \frac{\partial i_2}{\partial x_1} \end{bmatrix}\\
&= \begin{bmatrix} 1_{i_1 \geq i_2} & 1_{i_1 < i_2} \end{bmatrix} \begin{bmatrix} \frac{\partial i_1}{\partial x_1} \\ \frac{\partial i_2}{\partial x_1} \end{bmatrix}\\
&= \left(\frac{\partial i_1}{\partial x_1}\right)_{i_1 \geq i_2} + \left(\frac{\partial i_2}{\partial x_1}\right)_{i_1 < i_2}
\end{align*}

Similarly, we have:
\begin{align*}
\frac{\partial o_{min}}{\partial x_1} &= \frac{\partial o_{min}}{\partial i} \frac{\partial i}{\partial x_1}\\
&= \begin{bmatrix} \frac{\partial o_{min}}{\partial i_1} & \frac{\partial o_{min}}{\partial i_2} \end{bmatrix} \begin{bmatrix} \frac{\partial i_1}{\partial x_1} \\ \frac{\partial i_2}{\partial x_1} \end{bmatrix}\\
&= \begin{bmatrix} 1_{i_1 < i_2} & 1_{i_1 \geq i_2} \end{bmatrix} \begin{bmatrix} \frac{\partial i_1}{\partial x_1} \\ \frac{\partial i_2}{\partial x_1} \end{bmatrix}\\
&= \left(\frac{\partial i_1}{\partial x_1}\right)_{i_1 < i_2} + \left(\frac{\partial i_2}{\partial x_1}\right)_{i_1 \geq i_2}
\end{align*}

Finally, we have:
\begin{align*}
\frac{\partial E}{\partial x_1} &= \frac{\partial E}{\partial y} \frac{\partial (o_{max} + o_{min})}{\partial x_1}\\
&= \frac{\partial E}{\partial y} \left( \frac{\partial o_{max}}{\partial x_1} + \frac{\partial o_{min}}{\partial x_1} \right)\\
&= \frac{\partial E}{\partial y} \left( \left(\frac{\partial i_1}{\partial x_1}\right)_{i_1 \geq i_2} + \left(\frac{\partial i_2}{\partial x_1}\right)_{i_1 < i_2} + \left(\frac{\partial i_1}{\partial x_1}\right)_{i_1 < i_2} + \left(\frac{\partial i_2}{\partial x_1}\right)_{i_1 \geq i_2} \right)\\
&= \frac{\partial E}{\partial y} \left( \frac{\partial i_1}{\partial x_1} + \frac{\partial i_2}{\partial x_1} \right) \tag{Only one of $i_1 \geq i_2$ or $i_1 < i_2$ can be true}\\
&= \frac{\partial E}{\partial y} \frac{\partial i_1}{\partial x_1} \tag{$i_2$ does not depend on $x_1$}\\
&= \frac{\partial E}{\partial y} \cdot \frac{\partial}{\partial x_1} \frac{1}{1+e^{-x_1}}\\
&= \frac{\partial E}{\partial y} \frac{e^{-x_1}}{(1+e^{-x_1})^2}\\
&= \frac{\partial E}{\partial y} \frac{e^{x_1}}{(e^{x_1}+1)^2}\\
\end{align*}

From the network structure, we can see that the partial derivatives of $o_{max}$ and $o_{min}$ and with respect to $x_2$ would be the same as that with respect to $x_1$, since all the dependencies are similar beyond that layer.

Thus, we have:
\begin{align*}
\frac{\partial o_{max}}{\partial x_2} &= \left(\frac{\partial i_1}{\partial x_2}\right)_{i_1 \geq i_2} + \left(\frac{\partial i_2}{\partial x_2}\right)_{i_1 < i_2}\\
\frac{\partial o_{min}}{\partial x_2} &= \left(\frac{\partial i_1}{\partial x_2}\right)_{i_1 < i_2} + \left(\frac{\partial i_2}{\partial x_2}\right)_{i_1 \geq i_2}
\end{align*}

And similar to above, we get:
\begin{align*}
\frac{\partial E}{\partial x_2} &= \frac{\partial E}{\partial y} \left( \left(\frac{\partial i_1}{\partial x_2}\right)_{i_1 \geq i_2} + \left(\frac{\partial i_2}{\partial x_2}\right)_{i_1 < i_2} + \left(\frac{\partial i_1}{\partial x_2}\right)_{i_1 < i_2} + \left(\frac{\partial i_2}{\partial x_2}\right)_{i_1 \geq i_2} \right)\\
&= \frac{\partial E}{\partial y} \left( \frac{\partial i_1}{\partial x_2} + \frac{\partial i_2}{\partial x_2} \right) \tag{Only one of $i_1 \geq i_2$ or $i_1 < i_2$ can be true}\\
&= \frac{\partial E}{\partial y} \frac{e^{x_2}}{(e^{x_2}+1)^2}
\end{align*}

\end{solution}
\fi


\item[1.2] \textbf{Batch Normalization}
\ifnum\me<2
\begin{solution}\\
TODO
\end{solution}
\fi


\end{enumerate}


\printbibliography
\end{document}