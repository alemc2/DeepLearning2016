%me=0 student solutions (ps file), me=1 - my solutions (sol file), me=2 - assignment (hw file)
\def\me{0}
\def\num{1}  %homework number
\def\due{Thursday, March 10}  %due date
\def\course{DS-GA.1008 Deep Learning} %course name, changed only once
\def\name{R2DEEP2 (Ankit Vani, Srivas Venkatesh)}   %student changes (instructor keeps!)
%
\iffalse
INSTRUCTIONS: replace # by the homework number.
(if this is not ps#.tex, use the right file name)

  Clip out the ********* INSERT HERE ********* bits below and insert
appropriate TeX code.  Once you are done with your file, run

  ``latex ps#.tex''

from a UNIX prompt.  If your LaTeX code is clean, the latex will exit
back to a prompt.  To see intermediate results, type

  ``xdvi ps#.dvi'' (from UNIX prompt)
  ``yap ps#.dvi'' (if using MikTex in Windows)

after compilation. Once you are done, run

  ``dvips ps#.dvi''

which should print your file to the nearest printer.  There will be
residual files called ps#.log, ps#.aux, and ps#.dvi.  All these can be
deleted, but do not delete ps1.tex. To generate postscript file ps#.ps,
run

  ``dvips -o ps#.ps ps#.dvi''

I assume you know how to print .ps files (``lpr -Pprinter ps#.ps'')
\fi
%
\title{Deep Learning}
\documentclass[11pt]{article}
\usepackage{amsfonts,amsmath,physics}
\usepackage[backend=bibtex,sorting=none]{biblatex}
\usepackage{latexsym, graphicx}
\usepackage{amssymb, gensymb}
\usepackage{mathtools}
\usepackage{clrscode3e}
\usepackage{longtable}
\usepackage{tikz}
\usepackage{bm}
\usepackage{hyperref}
\usetikzlibrary{trees}
\usepackage{tikz-qtree}
\usepackage{graphicx,float}
\setlength{\oddsidemargin}{.0in}
\setlength{\evensidemargin}{.0in}
\setlength{\textwidth}{6.5in}
\setlength{\topmargin}{-0.4in}
\setlength{\textheight}{8.5in}

\addbibresource{bibliography.bib}
\graphicspath{ {../images/} }

\newcommand{\handout}[5]{
   \renewcommand{\thepage}{#1, Page \arabic{page}}
   \noindent
   \begin{center}
   \framebox{
      \vbox{
    \hbox to 5.78in { {\bf \course} \hfill #2 }
       \vspace{4mm}
       \hbox to 5.78in { {\Large \hfill #5  \hfill} }
       \vspace{2mm}
       \hbox to 5.78in { {\it #3 \hfill #4} }
      }
   }
   \end{center}
   \vspace*{4mm}
}

\newcommand{\LCA}{\mbox{\sf LCA}}

\newcommand{\rs}{\rightsquigarrow}
\newcommand{\ls}{\leftsquigarrow}

\newcounter{pppp}
\newcommand{\prob}{\arabic{pppp}}  %problem number
\newcommand{\increase}{\addtocounter{pppp}{1}}  %problem number

%first argument desription, second number of points
\newcommand{\newproblem}[1]{
\ifnum\me=0
\ifnum\prob>0 \newpage \fi
\increase
\setcounter{page}{1}
\handout{\name, Assignment \num, Section \arabic{pppp}}{\today}{Team: \name}{Due:
\due}{Solutions to Assignment \num}
\section*{Problem \prob~ - #1 \hfill}
\else
\increase
\section*{Problem \num-\prob~ - #1 \hfill}
\fi
}

%\newcommand{\newproblem}[2]{\increase
%\section*{Problem \num-\prob~(#1) \hfill {#2}}
%}

\def\squarebox#1{\hbox to #1{\hfill\vbox to #1{\vfill}}}
\def\qed{\hspace*{\fill}
        \vbox{\hrule\hbox{\vrule\squarebox{.667em}\vrule}\hrule}}
\newenvironment{solution}{\begin{trivlist}\item[]{\bf Solution:}}
                      {\qed \end{trivlist}}
\newenvironment{solsketch}{\begin{trivlist}\item[]{\bf Solution Sketch:}}
                      {\qed \end{trivlist}}
\newenvironment{code}{\begin{tabbing}
12345\=12345\=12345\=12345\=12345\=12345\=12345\=12345\= \kill }
{\end{tabbing}}

%%%%%\newcommand{\eqref}[1]{Equation~(\ref{eq:#1})}

\newcommand{\hint}[1]{({\bf Hint}: {#1})}
%Put more macros here, as needed.
\newcommand{\room}{\medskip\ni}
\newcommand{\brak}[1]{\langle #1 \rangle}
\newcommand{\bit}[1]{\{0,1\}^{#1}}
\newcommand{\zo}{\{0,1\}}
\newcommand{\C}{{\cal C}}

\newcommand{\nin}{\not\in}
\newcommand{\set}[1]{\{#1\}}
\renewcommand{\ni}{\noindent}
\renewcommand{\gets}{\leftarrow}
\renewcommand{\to}{\rightarrow}
\newcommand{\assign}{:=}
\newcommand{\cT}{\mathcal{T}}

\DeclareMathOperator*{\E}{E}
\DeclareMathOperator*{\argmin}{arg\,min}
\DeclareMathOperator*{\argmax}{arg\,max}
\DeclareMathOperator*{\ORA}{\vee}
\newcommand{\R}{\mathbb{R}}

%\DeclarePairedDelimiter{\abs}{\lvert}{\rvert}
%\DeclarePairedDelimiter{\norm}{\lVert}{\rVert}
\DeclarePairedDelimiter{\inprod}{\langle}{\rangle}

\newcommand\Perm[2][n]{\prescript{#1\mkern-2.5mu}{}P_{#2}}

\newcommand{\AND}{\wedge}
\newcommand{\OR}{\vee}

\newcommand{\mexp}{\mathrm{e}}

\makeatletter
\newtoks\@tabtoks
\newcommand\addtabtoks[1]{\global\@tabtoks\expandafter{\the\@tabtoks#1}}
\newcommand\eaddtabtoks[1]{\edef\mytmp{#1}\expandafter\addtabtoks\expandafter{\mytmp}}
\newcommand*\resettabtoks{\global\@tabtoks{}}
\newcommand*\printtabtoks{\the\@tabtoks}
\makeatother

\allowdisplaybreaks


\begin{document}


\newproblem{More Backpropagation}


\begin{enumerate}

\item[1.1] \textbf{Backpropagation through a DAG of modules}
\ifnum\me<2
\begin{solution}\\
Let $o_{max}$ be the output of the \texttt{max} module, and $o_{min}$ be the output of the \texttt{min} module.

Let $i_1$ be the output of the first \texttt{sigmoid} module, and $i_2$ be the output of the second \texttt{sigmoid} module. Furthermore, let $i = \begin{bmatrix} i_1 \\ i_2 \end{bmatrix}$.

We have:
\begin{align*}
\frac{\partial o_{max}}{\partial x_1} &= \frac{\partial o_{max}}{\partial i} \frac{\partial i}{\partial x_1}\\
&= \begin{bmatrix} \frac{\partial o_{max}}{\partial i_1} & \frac{\partial o_{max}}{\partial i_2} \end{bmatrix} \begin{bmatrix} \frac{\partial i_1}{\partial x_1} \\ \frac{\partial i_2}{\partial x_1} \end{bmatrix}\\
&= \begin{bmatrix} 1_{i_1 \geq i_2} & 1_{i_1 < i_2} \end{bmatrix} \begin{bmatrix} \frac{\partial i_1}{\partial x_1} \\ \frac{\partial i_2}{\partial x_1} \end{bmatrix}\\
&= \left(\frac{\partial i_1}{\partial x_1}\right)_{i_1 \geq i_2} + \left(\frac{\partial i_2}{\partial x_1}\right)_{i_1 < i_2}
\end{align*}

Similarly, we have:
\begin{align*}
\frac{\partial o_{min}}{\partial x_1} &= \frac{\partial o_{min}}{\partial i} \frac{\partial i}{\partial x_1}\\
&= \begin{bmatrix} \frac{\partial o_{min}}{\partial i_1} & \frac{\partial o_{min}}{\partial i_2} \end{bmatrix} \begin{bmatrix} \frac{\partial i_1}{\partial x_1} \\ \frac{\partial i_2}{\partial x_1} \end{bmatrix}\\
&= \begin{bmatrix} 1_{i_1 < i_2} & 1_{i_1 \geq i_2} \end{bmatrix} \begin{bmatrix} \frac{\partial i_1}{\partial x_1} \\ \frac{\partial i_2}{\partial x_1} \end{bmatrix}\\
&= \left(\frac{\partial i_1}{\partial x_1}\right)_{i_1 < i_2} + \left(\frac{\partial i_2}{\partial x_1}\right)_{i_1 \geq i_2}
\end{align*}

Finally, we have:
\begin{align*}
\frac{\partial E}{\partial x_1} &= \frac{\partial E}{\partial y} \frac{\partial (o_{max} + o_{min})}{\partial x_1}\\
&= \frac{\partial E}{\partial y} \left( \frac{\partial o_{max}}{\partial x_1} + \frac{\partial o_{min}}{\partial x_1} \right)\\
&= \frac{\partial E}{\partial y} \left( \left(\frac{\partial i_1}{\partial x_1}\right)_{i_1 \geq i_2} + \left(\frac{\partial i_2}{\partial x_1}\right)_{i_1 < i_2} + \left(\frac{\partial i_1}{\partial x_1}\right)_{i_1 < i_2} + \left(\frac{\partial i_2}{\partial x_1}\right)_{i_1 \geq i_2} \right)\\
&= \frac{\partial E}{\partial y} \left( \frac{\partial i_1}{\partial x_1} + \frac{\partial i_2}{\partial x_1} \right) \tag{Only one of $i_1 \geq i_2$ or $i_1 < i_2$ can be true}\\
&= \frac{\partial E}{\partial y} \frac{\partial i_1}{\partial x_1} \tag{$i_2$ does not depend on $x_1$}\\
&= \frac{\partial E}{\partial y} \cdot \frac{\partial}{\partial x_1} \frac{1}{1+e^{-x_1}}\\
&= \frac{\partial E}{\partial y} \frac{e^{-x_1}}{(1+e^{-x_1})^2}\\
&= \frac{\partial E}{\partial y} \frac{e^{x_1}}{(e^{x_1}+1)^2}\\
\end{align*}

From the network structure, we can see that the partial derivatives of $o_{max}$ and $o_{min}$ and with respect to $x_2$ would be the same as that with respect to $x_1$, since all the dependencies are similar beyond that layer.

Thus, we have:
\begin{align*}
\frac{\partial o_{max}}{\partial x_2} &= \left(\frac{\partial i_1}{\partial x_2}\right)_{i_1 \geq i_2} + \left(\frac{\partial i_2}{\partial x_2}\right)_{i_1 < i_2}\\
\frac{\partial o_{min}}{\partial x_2} &= \left(\frac{\partial i_1}{\partial x_2}\right)_{i_1 < i_2} + \left(\frac{\partial i_2}{\partial x_2}\right)_{i_1 \geq i_2}
\end{align*}

And similar to above, we get:
\begin{align*}
\frac{\partial E}{\partial x_2} &= \frac{\partial E}{\partial y} \left( \left(\frac{\partial i_1}{\partial x_2}\right)_{i_1 \geq i_2} + \left(\frac{\partial i_2}{\partial x_2}\right)_{i_1 < i_2} + \left(\frac{\partial i_1}{\partial x_2}\right)_{i_1 < i_2} + \left(\frac{\partial i_2}{\partial x_2}\right)_{i_1 \geq i_2} \right)\\
&= \frac{\partial E}{\partial y} \left( \frac{\partial i_1}{\partial x_2} + \frac{\partial i_2}{\partial x_2} \right) \tag{Only one of $i_1 \geq i_2$ or $i_1 < i_2$ can be true}\\
&= \frac{\partial E}{\partial y} \frac{e^{x_2}}{(e^{x_2}+1)^2}
\end{align*}

\end{solution}
\fi


\item[1.2] \textbf{Batch Normalization}
\begin{enumerate}
\item[1.2.1] Given $\frac{\partial E}{\partial y_k}$ write down $\frac{\partial E}{\partial x_k}$, where E is the energy function.
\ifnum\me<2
\begin{solution}
\\For this problem we are going to consider the notation used in the paper \cite{BN} as the question is ill-formed.
That is we are given the function 
\begin{align*}
y^{(k)} &= \frac{x^{(k)} - E(x^{(k)})}{\sqrt{(\sigma(x^{(k)}))^2}}
\end{align*}
where $k$ is a dimension of the $d$ dimensional input and we take these mean and variance along the batch inputs on those dimensions. Now assuming the mini-batch size to be $m$, that is we have $m$ samples, we get the following along a particular dimension:
\begin{align*}
y_i &= \frac{x_i-E(x_i)}{\sqrt{(\sigma(x_i))^2}}
\end{align*}
where $i$ ranges from $1 \ldots m$. This is the same across all the dimensions and the normalization of each dimension is independent of the other. Then we can write the gradient of the energy function wrt. $x^{(k)}_i$ as follows:
\begin{equation}
\frac{\partial E}{\partial x^{(k)}_i} = \sum_{j=1}^m \frac{\partial E}{\partial y^{(k)}_j} \cdot \frac{\partial y^{(k)}_j}{\partial x^{(k)}_i}
\label{eq1}
\end{equation}
Dropping the superscripts for convenience (we will later get it back by showing it as vector elements), we see the following:
\begin{equation}
\frac{\partial y_j}{\partial x_i} =
\begin{cases}
-\frac{1}{m\sigma} - \frac{(x_j-E(x)) (x_i-E(x))}{m\sigma^3} & j \neq i\\
(1-\frac{1}{m})\cdot \frac{1}{\sigma} - \frac{(x_i-E(x))^2}{m\sigma^3} & j = i
\end{cases}
\label{eq2}
\end{equation}
The above is arrived at by differentiation by parts and using the fact that $(\sigma(x))^2 = E(x^2) - E(x)^2$. Putting back the superscripts we get:
\begin{equation}
\frac{\partial E}{\partial x^{(k)}} =
\begin{bmatrix}
\frac{\partial E}{\partial x^{(1)}_1} & \frac{\partial E}{\partial x^{(1)}_2} & \cdots & \frac{\partial E}{\partial x^{(1)}_m}\\\\
\frac{\partial E}{\partial x^{(2)}_1} & \frac{\partial E}{\partial x^{(2)}_2} & \cdots & \frac{\partial E}{\partial x^{(2)}_m}\\\\
\cdots & \cdots & \cdots & \cdots \\\\
\frac{\partial E}{\partial x^{(d)}_1} & \frac{\partial E}{\partial x^{(d)}_2} & \cdots & \frac{\partial E}{\partial x^{(d)}_m}
\end{bmatrix}
\label{eq3}
\end{equation}
Substituting results from (\ref{eq1}),(\ref{eq2}) in (\ref{eq3}), we get what the question asks.
\end{solution}
\fi

\item[1.2.2] Can you derive
a way to incorporate a learnable shift variable into the current formula? Let’s assume
the considered variable is denoted by $\epsilon$. Please write down the forward and backward
pass, and derive $\frac{\partial E}{\partial \epsilon}$

\ifnum\me<2
\begin{solution}
\\We can do this by changing the batch normalization function as follows:
$$y^{(k)} = \frac{x^{(k)} - E(x^{(k)})}{\sqrt{(\sigma(x^{(k)}))^2}} + \epsilon^{(k)}$$
That is we introduce a learnable shift to our normalization. With that the gradient is simply computed as:
$$\frac{\partial E}{\partial \epsilon^{(k)}} =
\begin{bmatrix}
\frac{\partial E}{\partial y^{(k)}_1} & \frac{\partial E}{\partial y^{(k)}_2} & \cdots & \frac{\partial E}{\partial y^{(k)}_m}
\end{bmatrix}
\cdot
\begin{bmatrix}
1 \\ 1 \\ \cdots \\ 1
\end{bmatrix}
= \sum_{i=1}^{m} \frac{\partial E}{\partial y^{(k)}_i}
$$
Thus we can write the required gradient in the form
$$\frac{\partial E}{\partial \epsilon} =
\begin{bmatrix}
\sum_{i=1}^{m} \frac{\partial E}{\partial y^{(1)}_i}\\
\sum_{i=1}^{m} \frac{\partial E}{\partial y^{(2)}_i}\\
\cdots\\
\sum_{i=1}^{m} \frac{\partial E}{\partial y^{(d)}_i}
\end{bmatrix}$$
\end{solution}
\fi
\end{enumerate}
\end{enumerate}

\newproblem{STL-10: semi-supervised image recognition}
\ifnum\me<2
\begin{solution}
\\We performed our experiments on the STL-10 dataset \cite{coates2011analysis} to classify real world images into 10 classes: airplane, bird, car, cat, deer, dog, horse, monkey, ship, truck.
\\Since STL-10 is a dataset with a much larger corpus of unlabeled data, we used ideas from \cite{conv_kmeans} to pre-train first 4 layers of our network using the unsupervised data. We also use various augmentations and a multi-scale architecture to generalize better. More details are presented below:
\begin{enumerate}
\item Overview of the network
\begin{itemize}
\item Dataset, partitioning and pre-processing:
\\STL-10 dataset has 10 classes with 500 training images per class and 800 test images per class. We had a set of 100 images per class split off from the training images which was used for validation. We thus had 4000 training images, 1000 validation images and 8000 testing images. Besides these labeled data we also have 100,000 unlabeled samples.
\\The loaded training data is converted to YUV colorspace. Spatial contrast normalization is applied to Y component of each image and global normalization to have zero mean and unit variance is performed on the U,V channels. The parameters
used for this normalization are stored and used to normalize the validation and test set as well.
\\The images are converted back to RGB before being passed through our model as the pretraining (which is described later) is done using RGB images.

\item Model:
The model we use to train this network is 
%Talk about each level structure

%Talk about 3 scale levels

%Talk about how they are weighed and mean

\item Augmentations:
We perform various data augmentations in order to generalize the network to learn important features invariant of the object classes. These augmentations have been tried and tested in works such as \cite{exemplar} \cite{baidu}. The various augmentations we perform are:
\begin{enumerate}
\item Translation: Vertical and horizontal translation by -15 to +15 pixels.
\item Scaling, Stretching: Scaling height and width independently by 0.75 to 1.25 times.
\item Rotation: Rotation of the image by an angle of -18 to 18 degrees.
\item Contrast Shifting: raise saturation and value (S and V components of the HSV color representation)
of all pixels to a power between 0.25 and 2 (same for all pixels within a patch), multiply
these values by a factor between 0.7 and 1.4, add to them a value between −0.1 and 0.1 as done in \cite{exemplar}
\item Color Shifting: Altering R,G,B channels independently by -30 to +30 as done in \cite{baidu}
\end{enumerate}
We also fill the black spaces created by these transforms (if any) by the mean color at the edges to avoid abrupt color transition. Some of these random transformations applied to a single image are shown below (the combined random transform is applied 50 times to possibly see all the above transforms).
\begin{figure}[H]
\centering
\foreach \x in {1,...,51}
{
	\includegraphics[scale=0.2]{lizard\x.png}
	\ifnum\x=17
	\\
	\fi
	\ifnum\x=34
	\\
	\fi
}
\\
\caption{Random transforms applied 50 times to this image to view all possible transforms. First image is the original image.}
\end{figure}


\end{itemize}

\item Visualizations:
We try to visualize some of the layers and activations to see how our network has trained. Some of them are shown below:
\begin{itemize}
\item First layer filters learnt through clustering:
\\The 3x3 filers learnt for the first layer by clustering on unlabeled data is shown below.
\begin{figure}[H]
\centering
\foreach \x in {1,...,64}
{
	\includegraphics[scale=0.5]{filter\x.png}
	\ifnum\x=16
	\\
	\fi
	\ifnum\x=32
	\\
	\fi
	\ifnum\x=48
	\\
	\fi
}
\\
\caption{Filters created by clustering on unsupervised data. These are actually 3x3 filters which are enlarged and enhanced for better display}
\end{figure}
The above filters are actually 3x3 in size but are enlarged and enhanced for better display. This might cause it to look a little saturated and diffused at places than what it actually was. As we can see form this figure the learned filters are detecting some prominent edges just as we had expected. Another point to note is that since we are doing convolutional K-means clustering as described in \cite{conv_kmeans} we have avoided generating similar filters.

\item t-SNE:
\\We used t-SNE to cluster the representation learnt by our network on the second last layer. It did cluster quite well with meaningful groups and neighbours indicating that the representations learnt by our network are quite well in line with the expected classes.
\end{itemize}
\item Experiments and Observations:

\end{enumerate}
\end{solution}
\fi


\printbibliography
\end{document}